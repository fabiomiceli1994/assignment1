\documentclass{article}
%\usepackage[version=3]{mhchem} % Package for chemical equation typesetting
%\usepackage{siunitx} % Provides the \SI{}{} and \si{} command for typesetting SI units
%\usepackage{graphicx} % Required for the inclusion of images
%\usepackage{natbib} % Required to change bibliography style to APA
\usepackage{amsmath} % Required for some math elements 
\usepackage[margin=1in]{geometry}
\usepackage{microtype}
\usepackage[english]{babel}
\usepackage[utf8]{inputenx}
\usepackage{float}
\usepackage{graphics}
\usepackage{graphicx}
\usepackage{subfigure}
\usepackage{amsmath}
\usepackage{textcomp}
\usepackage{makeidx}
\usepackage{hyperref}
\usepackage{braket}
%\usepackage[latin1]{inputenc}
\usepackage{amsthm}
\usepackage{amsfonts}
\usepackage{amssymb}
\usepackage{graphicx}
\usepackage{fullpage}
\usepackage{hyphenat}
\usepackage{float}
\usepackage{longtable}
\usepackage{picture}
\usepackage{multicol}
\usepackage{fancyhdr}
\usepackage{wrapfig}
\usepackage{geometry}

\theoremstyle{theorem}
\newtheorem{theorem}{Theorem}
\newtheorem{proposition}{Proposition}

\theoremstyle{definition}
\newtheorem{definition}{Definition}



%\usepackage{tikz}
%\usetikzlibrary{graphs}
%\usepackage{pgfplots}
%\pgfplotsset{compat=newest}

\setlength\parindent{0pt} % Removes all indentation from paragraphs

\renewcommand{\figurename}{Grafico}
\renewcommand{\tablename}{Tabella}
\newcommand{\at}[2][]{#1\Big|_{#2}}
\renewcommand{\labelenumi}{\alph{enumi}.} % Make numbering in the enumerate environment by letter rather than number (e.g. section 6)

%\usepackage{times} % Uncomment to use the Times New Roman font

%----------------------------------------------------------------------------------------
%	DOCUMENT INFORMATION
%----------------------------------------------------------------------------------------

\title{C1 - Assignment 1 Report: Sparse Matrices.} % Title

\author{Student Number: 1894945} % Author name

\date{\today} % Date for the report

\begin{document}

\maketitle % Insert the title, author and date

\begin{center}
C1 - Assignment 1 Report \hfill
Student Number: 1894945
\vspace{3pt} \hrule \vspace{3pt} \hrule
\end{center}

%\clearpage
\tableofcontents

\clearpage
% If you wish to include an abstract, uncomment the lines below
\begin{abstract}


\end{abstract}
\clearpage 

\section{Introduction}

\subsection{Well-posed, direct problems}
The problems that will be addressed in the following are assumed to be always representable in the form: 

\begin{equation}
	\label{eqn:general-problem}
	F(x, d) = 0
\end{equation}

where $x$ represents the unknown, $d$ the set of data from which the solution depends on and $F$ the functional relation between $x$ and $d$. Such types of problem are called \emph{direct problems} (\cite{numerical-math}).\\
If the problem admits a unique solution\footnote{In this case $d$ is said to be admissible for \ref{eqn:general-problem}.} $x$ that depends continuously on the data $d$, then the problem is said to be \emph{well-posed} or \emph{stable}. Whenever the aforementioned properties are not satisfied, the problem is said to be \emph{ill-posed}.\\

\subsection{Numerical Methods}
In the following, it will always be assumed that problem \ref{eqn:general-problem} is well-posed. A numerical method for the approximate solution of the aforementioned equation consists in a sequence of approximate problems:

\begin{equation}
	\label{eqn:num-method}
	F_n(x_n, d_n) = 0 \hspace{3mm} n\ge 1
\end{equation}

with the underlying expectation that $x_n\rightarrow x$ as $n\rightarrow\infty$, i.e. the approximate solution converges to the exact one. 

\begin{definition}
The numerical method \ref{eqn:num-method} is convergent iff

$$\forall\epsilon >0,\hspace{1mm} \exists n_\epsilon,\hspace{1mm} \exists\delta(n_\epsilon)\hspace{3mm} | \hspace{3mm}\forall n > n_\epsilon,\hspace{1mm} \forall\delta d_n : ||x(d)-x_n(d+\delta d_n)||<\epsilon$$

where $d_n$ is an admissible datum for the $n^{th}$ approximate problem, $\delta d_n$ a perturbation of $d_n$, $x_n(d+\delta d_n)$ the corresponding solution of it and $x(d)$ the solution for corresponding
 exact problem.
\end{definition}



\section{Problem setup}
\subsection{Performed tests}

\section{Conclusive remarks}


\cleardoublepage
%\add1contentsline{toc}{chapter}{\bibname}
\begin{thebibliography}{99}

\bibitem{numerical-math} A. Quateroni, R. Sacco, F. Saleri;
\emph{Numerical Mathematics}, Vol.37, Springer Verlag, (2007).







\printindex
\end{thebibliography}
\bibliography{bibliography} % BibTeX database without .bib extension
\end{document}



%----------------------------------------------------------------------------------------
%	BIBLIOGRAPHY
%----------------------------------------------------------------------------------------

%\bibliographystyle{apalike}

%\bibliography{sample}

----------------------------------------------------------------------------------------


%end{document}